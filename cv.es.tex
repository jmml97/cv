%%%%%%%%%%%%%%%%%%%%%%%%%%%%%%%%%%%%%%%%%
% Compact Academic CV
% LaTeX Template
% Version 2.0 (6/7/2019)
%
% This template originates from:
% https://www.LaTeXTemplates.com
%
% Authors:
% Dario Taraborelli (http://nitens.org/taraborelli/home)
% Vel (vel@LaTeXTemplates.com)
%
% Modified by:
% José María Martín Luque (https://jmml.me)
%
% License:
% CC BY-NC-SA 3.0 (http://creativecommons.org/licenses/by-nc-sa/3.0/)
%
%%%%%%%%%%%%%%%%%%%%%%%%%%%%%%%%%%%%%%%%%

%----------------------------------------------------------------------------------------
%	PACKAGES AND OTHER DOCUMENT CONFIGURATIONS
%----------------------------------------------------------------------------------------

\documentclass[11pt]{article} % Default document font size
\usepackage[spanish]{babel}
\usepackage[utf8]{inputenc}

\input{structure.tex} % Include the file specifying the document structure and styling

% Set PDF meta-information
\hypersetup{
	pdftitle={José María Martín Luque - Curriculum vitae},
	pdfauthor={José María Martín Luque}
}

\usepackage{fancyhdr}
\fancyhf{}
\renewcommand{\headrulewidth}{0pt}
\cfoot{
	\scriptsize
	Última actualización: \today
}

%----------------------------------------------------------------------------------------

\begin{document}

\pagestyle{fancy}

%----------------------------------------------------------------------------------------
%	CONTACT AND GENERAL INFORMATION
%----------------------------------------------------------------------------------------

{\LARGE\bfseries José María Martín Luque}
\bigskip\bigskip\medskip

\ifdefined\emailphone
  \input{email-phone.es.tex}
\fi
Página web: \href{https://jmml.me}{jmml.me}\\
Github: \href{https://github.com/jmml97}{@jmml97}\\
LinkedIn: \href{https://linkedin.com/in/jmml97}{@jmml97}\\

%----------------------------------------------------------------------------------------
%	EDUCATION
%----------------------------------------------------------------------------------------

\section*{Educación}

\years{2015-2020} Grado en Ingeniería Informática, Universidad de Granada\\
	{\color{gray}\itshape Enfocado en el desarrollo web y la administración de sistemas}\\

\years{2015-2020} Grado en Matemáticas, Universidad de Granada\\


%----------------------------------------------------------------------------------------
%	PROJECTS ACTIVITIES
%----------------------------------------------------------------------------------------

\section*{Proyectos y otras actividades}

\years{2019-2020} Trabajo de fin de grado\\
{\color{gray}El objetivo del trabajo era estudiar e implementar en SageMath el algoritmo de Peterson-Gorenstein-Zierler para códigos cíclicos sesgados. El texto y el código están disponibles en \href{https://github.com/jmml97/tfg}{GitHub}.}\\
{\color{gray}\itshape — Calificado con matrícula de honor}\\

\years{2018-2019} Director de Comunicación de la Delegación General de Estudiantes de la Universidad de Granada\\
{\color{gray} Responsable del diseño y la ejecución de campañas de comunicación dirigidas a 47~000 estudiantes. A cargo también de las redes sociales y las relaciones con la prensa local.}\\

Colaborador en \href{https://github.com/libreim/apuntesdgiim}{apuntesDGIIM}\\ 
{\color{gray} Proyecto estudiantil dedicado a la creación de libros de texto gratuitos y de código abierto para asignaturas de Informática y Matemáticas.

Además de colaborar en la elaboración de los textos, creé las plantillas utilizadas, además de la página web del proyecto, así como de su compilación y distribución.}\\

Desarrollo web\\
{\color{gray} He diseñado y desarrollado: \href{https://cielos.es}{cielos.es}, \href{https://herminialuque.com}{herminialuque.com} y mi \href{https://jmml.me}{página web personal}.}\\

%----------------------------------------------------------------------------------------
%	LANGUAGES & SKILLS
%----------------------------------------------------------------------------------------
\vspace{1cm}

\begin{minipage}[t]{0.5\textwidth}
	\section*{Idiomas}

	\begin{tabular}{@{}ll}
	\textsc{Castellano:} & Lengua materna\\

	\textsc{Inglés:} & Fluido (B2)\\

	\textsc{Francés:} & Básico\\
	\end{tabular}\\
\end{minipage}
\begin{minipage}[t]{0.5\textwidth}
	\section*{Habilidades}

	\begin{tabular}{@{}ll}
	\textsc{Programación:} & C++, Python, Swift\\

	\textsc{Desarrollo web:} & HTML, CSS, PHP, Javascript\\

	\textsc{Otras:} & Docker, MySQL, \LaTeX\\

	\end{tabular}\\
\end{minipage}

\end{document}
